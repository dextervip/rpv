\documentclass{abnt}
\usepackage[utf8]{inputenc}
\usepackage[english,brazilian]{babel}
\usepackage{hyperref}
\usepackage{url}
\usepackage{indentfirst}
\usepackage{blindtext}
\hypersetup{%
    pdfborder = {0 0 0}
}
\usepackage{graphicx}
\graphicspath{{./imagens/}}
\usepackage{placeins}

\usepackage{longtable}
\usepackage{floatpag}% http://ctan.org/pkg/floatpag
\usepackage{fancyhdr}% http://ctan.org/pkg/fancyhdr
\usepackage[absolute]{textpos}

\usepackage{pdflscape}
\usepackage{rotating}

\usepackage{color, colortbl}
\definecolor{corImplementado}{rgb}{0.88,1,1}
\begin{document}

\autor{Rafael Tavares Amorim \\ Juliano Rodovalho Macedo \\ Wolmir L. F. Nemitz}

\titulo{Entrega 3: Gestão de Disponibilidade \\ Versão 2.0}

\instituicao{Universidade Federal do Pampa \par Engenharia da Software \par Resolução de Problemas V}

\local{Alegrete - Rio Grande do Sul, Brasil}

\data{15 de Setembro de 2012}

\capa

\folhaderosto

\tableofcontents

\listoffigures


\chapter{Introdução}

	Este relatório tem por finalidade apresentar a fundamentação teórica e tecnológica utilizada no terceiro \emph{sprint}, o nosso Product Backlog completo e atualizado, 
	o Sprint Backlog munido de nossas especificações de engenharia juntamente com o detalhamento da solução, e por fim,o link de referência visando o acesso ao sistema.

	\section{Metodologia Ágil}
	
		Metodologia ágil é um famoso conjunto de metodologias de desenvolvimento de software, 
		com foco na mitigação dos riscos, esta metodologia aposta em pequenos ciclos iterativos de desenvolvimento de software,
		onde ocorre a repetição de cada tarefa comum em um processo de software, ou seja, em cada ciclo realiza-se analise, projeto,
		implementação e testes, referentes aquele módulo, desta forma busca-se à capacidade real de implantar um nova versão 
		do sistema ao final de cada ciclo; Antes de iniciar-se uma nova iteração a equipe responsável por todo o projeto é reunida 
		e reavalia-se as características, metas e prioridades do projeto como um todo, para que estas sejam aplicadas no próximo ciclo.
		
		Os modelos ágeis são hoje a principal alternativa as metodologias tradicionais de desenvolvimento de software, 
		as quais apresentam diversos problemas estruturais que comprometem a maioria dos projetos, 
		baseados nas necessidades atuais do mercado de software.
	
	\section{Scrum}
	
		\subsection{Visão Geral}
		
			O \emph{scrum} é uma metodologia ágil de desenvolvimento de software, embora muitos consideram-no como um framework ágil, 
			o seus maior objetivo é agilizar o processo de desenvolvimento de software com pequenas entregas ao cliente de forma iterativa
			visando agregar-se valor ao produto a cada ciclo(iteração) . Em sua existem três papéis neste processo, são eles: \emph{Product Owner, Scrum Master, Team}.
		
		\subsection{Papéis}
		
			O \emph{Product Owner} é um papel muito importante no \emph{scrum}, cabe a ele definir quais funcionalidades devem ou
			não existir no sistema por meio de \emph{user stories} e também definir a prioridade delas. Este pa\texttt{}pel também
			trata de aprovar ou rejeitar o trabalho realizado e dizer se foi realizado como esperado.
			
			O \emph{scrum master} tem papel de manter a produtividade da equipe, facilitando o processo de \emph{scrum} e resolver impedimentos
			no decorrer do \emph{sprint}. Ele ajuda o time a se manter organizado e certificar que as regras do processo estão sendo
			seguidas corretamente. O \emph{scrum master} não dita o que o time deve fazer mas sim ajuda o time se manter autônomo,
			organizado e com uma boa produtividade. 
			
			O \emph{team} tem papel de transformar os itens de \emph{backlog} em produtos que serão entregues de forma incremental em cada
			sprint. Cabe ao time ter autonomia própria para se organizar e trabalhar de forma de colaborativa ajudando uns aos
			outros sem a necessidade de existência de um "chefe" para dizer o que deve ser feito.
		
		\subsection{Processo}
		
			O processo de \emph{scrum} é trabalhado em cima de vários ciclos chamados de \emph{sprint}, que possuem um tempo fixo, geralmente de
			1 a 4 semanas. Para cada \emph{sprint} é realizado a escolha de \emph{user stories} que estão no \emph{product backlog}. No \emph{product backlog} temos uma lista priorizada 
			pelo \emph{product owner} das funcionalidades desejadas do projeto, e parte desta lista irá forma o \emph{sprint backlog}. É uma coleção de \emph{user stories}.
			
			Em um ciclo, o time irá realizar o \emph{sprint backlog} que contêm os itens de \emph{backlog} que o time concordou 
			completar em um ciclo de \emph{sprint}, cada item de \emph{backlog} é quebrado em tarefas, atribuídas para equipe e estimado o tempo de realização.
			
					
			O \emph{backlog item} ou também conhecido como \emph{user story} é um recurso/funcionalidade desejada no sistema que vão compor o
			\emph{product backlog}. Uma \emph{user story} usualmente é composta por um ator do sistema e a ação para alcançar determinado
			objetivo. Exemplos: 
				\begin{itemize}
					\item Como um usuário Eu posso realizar a ação para alcançar determinado objetivo
					\item Como um cliente, eu posso criar um perfil no sistema
					\item Como um cliente, eu posso escolher um usuário e senha
					\item Como um administrador, eu posso bloquear contas de clientes
				\end{itemize}
					 
					
			No \emph{sprint planning}, a equipe decidirá quais \emph{backlog itens} irão para o \emph{sprint backlog}. Depois da priorização das \emph{user
			stories} pelo \emph{product owner}, a equipe seleciona o item do topo do \emph{product backlog}, o qual tem maior prioridade, para a
			discussão e criação das tarefas necessárias para entrega do item. Então as tarefas são atribuídas e estimadas em
			tempo, este processo é repetido até que o time decida que não há possibilidade de realização de mais itens.
			
			Com a ajuda do\emph{Burndown} que é um gráfico que oferece uma forma poderosa que podemos analisar o andamento do projeto. Podemos estimar a velocidade de andamento do time, atrasos e tempo de 
			execução de cada \emph{user story}.
			
			Por meio de um reunião diariamente, podemos também verificar impedimentos que podem afetar a produtividade da equipe, conhecida como \emph{daily meeting}, 
			é uma reunião com duração aproximada de 15 minutos,	conduzida pelo \emph{scrum master} e tem como foco 3 perguntas para todo membro da equipe:
				\begin{itemize}
					\item O que eu fiz ontem?
					\item O que eu estou fazendo hoje?
					\item O que está me impedindo de fazer?
				\end{itemize}
			
			Apesar da reunião ser conduzida pelo scrum master, cada membro deve responder para o time inteiro e não somente para o
			\emph{scrum master}, todos devem estar de pé na reunião e falar somente quando for perguntado pelo \emph{scrum master}. No final da
			reunião é discutido por todos soluções caso haja algum impedimento.
			
			No final de um \emph{sprint}, ocorre uma reunião informal, chamada \emph{sprint review} que tem intuito de demonstrar o que foi 
			realizado no sprint com a presença da equipe, \emph{scrum master}, \emph{product owner} e stakeholders. Também é feita a reunião chamada \emph{sprint retrospective}
			que é uma reunião da equipe que tem objetivo de discutir se o \emph{sprint} foi realizado com sucesso ou se é necessário melhorias.
			
			Um projeto em \emph{scrum} pode ser parado a qualquer momento, as vezes o cliente percebe que as funcionalidade implementadas já foram
			suficientes e atendem a necessidade atual assim não dispersando dinheiro em funcionalidade que talvez nunca seriam utilizadas.
			
			As informações desta seção foram retiradas da referência \cite{SCRUMEPF}.
	
		%Importação errada, arrumar futuramente
		
\begin{landscape}
	\chapter{Product Backlog}
		% Table generated by Excel2LaTeX from sheet 'Plan1'
		\begin{longtable}{|p{0.5in}|p{6in}|c|c|}
		\hline
		  {\bf ID} & {\bf Users Stories} & {\bf Tempo Previsto} & {\bf Status} \\
		\hline \rowcolor{corImplementado}
		         1 & Como professor, eu quero decidir quais compromissos são públicos ou privados para manter sigilo. &         10 & Implementado \\
		\hline \rowcolor{corImplementado}
		         2 & Como professor, eu quero a partir de uma disciplina cadastrar um plano de ensino com o programa e as bibliografias suplementares para as disciplinas que eu vou ministrar. &         40 & Implementado \\
		\hline \rowcolor{corImplementado}
		         3 & Como professor, eu quero cadastrar meus compromissos no sistema para que todos saibam quando não posso agendar outro compromisso. &         40 & Implementado \\
		\hline
		         4 & Como secretária eu desejo acessar as informações de horários e disponibilidade de cada sala do campus.  &         20 & Não Implementado \\
		\hline
		         5 & Como usuário, eu quero poder pedir minha senha por email para que eu possa recuperar caso eu tenha perdido. &         10 & Não Implementado \\
		\hline
		         6 & Como professor, eu quero navegar por todas as disciplinas disponíveis para visualizar as disciplinas que tenho interesse.  &         40 & Não Implementado \\
		\hline
		         7 & Como professor, eu quero ver a disponibilidade dos equipamentos como caixa de som, projetores e outros para poder fazer um agendamento. &         40 & Não Implementado \\
		\hline
		         8 & Como professor, eu quero visualizar minha grade horária de aulas para saber os horários que tenho que dar aula. &         40 & Não Implementado \\
		\hline \rowcolor{corImplementado}
		         9 & Como professor, eu quero informar os dias que estarei na unipampa para que os outros possam me encontrar para assuntos urgentes.  &         20 & Implementado \\
		\hline
		        10 & Como professor, eu quero informar meu horário disponível para as aulas, minhas disciplinas de preferência e área de formação para que o coordenador possa montar a grade de aulas de acordo com os dados que informei. &         40 & Parcialmente \\
		\hline
	
		        11 & Como professor, eu quero agendar reuniões com os outros professores em horários que não choquem com compromissos deles e esperar que eles aceitem.  &         30 & Não Implementado \\
		\hline
		        12 & Como usuário, eu quero ter um usuário e senha para que eu possa acessar minha conta pessoal no sistema. &         30 & Não Implementado \\
		\hline
		        13 & Como secretária, eu quero visualizar salas e equipamentos que estão alocados para quem e quando.  &         20 & Não Implementado \\
		\hline
		        14 & Como secretária, eu quero cadastrar os equipamentos de uso da instituição para que os professores possam ver sua disponibilidade.  &         40 & Não Implementado \\
		\hline
		        15 & Como secretária, eu quero cadastrar as salas, com seu tipo e capacidade para que o coordenador possa vê-las podendo organizar as ofertas das disciplinas. &         40 & Não Implementado \\
		\hline
		        16 & Como coordenador, eu quero cadastrar as disciplinas juntamente com sua descrição, ementa, pré-requisitos e sua necessidade de sala que poderão ser ofertadas no sistema e para que os professores possam escolher suas preferidas. &         40 & Não Implementado \\
		\hline
		        17 & Como coordenador, eu quero ver o perfil de cada professor para que eu possa avaliar suas áreas.  &         10 & Não Implementado \\
		\hline
		        18 & Como coordenador, eu quero alocar disciplinas em horários específicos, alocar os professores e salas para elas assim possibilitando sua oferta.  &         60 & Não Implementado \\
		\hline
		        19 & Como professor, eu quero me registrar no sistema para que eu possa fornecer meus dados ao coordenador. &         30 & Não Implementado \\
		\hline
		        20 & Refatorar formulário de cadastro de disciplinas e salas/tipos com intuito de melhorar a visibilidade e aparência.  &         40 & Não Implementado \\

		\hline
		        21 & Refatorar o DB de disciplinas separando curso, turma e campus em objetos diferentes e não atributos. &         30 & Não Implementado \\
		\hline
		        22 & Refatorar o DB de disciplina para receber pré-requisitos em forma de n objetos invés de campo texto. &         20 & Não Implementado \\
		\hline
		Defeito \#336 & Correção da marcação do Campo Obrigatório no form de disciplinas.  &          5 & Não Implementado \\
		\hline \rowcolor{corImplementado}
		Defeito \#399  & Correção da adição desconhecida da string (u'') ao editar um compromisso, no form de compromisso no área de professor. &          1 & Implementado \\
		\hline
		\caption{Product Backlog}
		\end{longtable}
		
		\textcolor{rltred}{$\bullet$}
		Cor - Implementado \\
		Cor - Não Implementado \\
		Cor - User Stories do Sprint Atual \\
		Data de geração: \today
		

\end{landscape}

		

\clearpage

\chapter{Sprint}
	
		Esta seção aborda as informações sobre o Sprint III, demonstrando o nosso \emph{Sprint Backlog} logo a baixo. Os diagramas de casos de uso e classes para cada \emph{User Story} está disponível nas seções seguintes.
	
			
		\section{Funcionalidades - Versão 2.0}
		
			As funcionalidades entregues nesta versão são:
		
			\begin{itemize} 
			\item Crud de Compromissos:
			\subitem Informar a frequência de repetição do compromisso
			\subitem Dias da semana para a repetição do compromisso
			\subitem Fim da repetição do compromisso
			
			\item Crud de Área de Formação:
			\subitem Informar área de formação
			\subitem Visualizar área de formação
			\subitem Remover área de formação
			
			\item Crud Disciplinas de Preferência:
			\subitem Informar disciplinas de preferência
			\subitem Visualizar disciplinas de preferência
			\subitem Editar disciplinas de preferência
			\subitem Remover disciplinas de preferência
			
			\item Crud Grade de Aulas:
			\subitem Visualizar grade de aulas
			
			\item Crud Disponibilidade Horária para Aulas:
			\subitem Informar disponibilidade de horário para ministrar aulas
			\subitem Visualizar disponibilidade de horário para ministrar aulas
			\subitem Remover disponibilidade de horário para ministrar aulas
			
			\end{itemize}
			
	\clearpage
	\section{User Story 1}
	
		\begin{itemize}
			\item Como professor, eu quero informar os dias que estarei na Unipampa para que os outros possam me encontrar para assuntos urgentes.
		\end{itemize}
		
		Status: Implementado
		
		A solução encontrada e combinada com P.O, seria implementar a recorrência de compromisso sendo assim o professor pode criar um compromisso para dizem que não está na unipampa ou vice-versa. A solução consiste em adicionar alguns campos no formulário para informar frequência(diária, semanal, mensal e anual), quando semanal(dias da semana) e data de fim da frequência.
		
		\subsection{Diagramas de Caso de Uso}	
		\begin{figure}[h]
			\begin{center}
				 \includegraphics[width=400px]{CasoUsoDiasAusencia}
				 \caption{Diagrama de Caso de Uso - Recorrência do Compromisso}
				 \label{fig:CasoUsoDiasAusencia}
			\end{center}
		\end{figure}
		
\begin{figure}[h]
			\begin{center}
				 \includegraphics[width=450px]{VisualizarDiasAusencia}
				 \caption{Caso de Uso - Visualizar Dias de Ausência}
				 %\label{fig:CasoUsoDisponibilidadeHorario}
			\end{center}
		\end{figure}
\begin{figure}[h]
			\begin{center}
				 \includegraphics[width=450px]{InformarDiasAusencia}
				 \caption{Caso de Uso - Informar Dias de Ausência}
				 %\label{fig:CasoUsoDisponibilidadeHorario}
			\end{center}
		\end{figure}
\begin{figure}[h]
			\begin{center}
				 \includegraphics[width=450px]{CancelarDiasAusencia}
				 \caption{Caso de Uso - Cancelar Dias de Ausência}
				 %\label{fig:CasoUsoDisponibilidadeHorario}
			\end{center}
		\end{figure}
		\FloatBarrier

		
		\subsection{Diagramas de Sequência}
		
		Não houve alterações no diagrama de sequência, apenas o diagrama de classes e caso de uso.
		
	
	\clearpage	
	\section{User Story 2}

		\begin{itemize}
			\item Como professor, eu quero informar meu horário disponível para as aulas, [Transferido para a Semana 03 (minhas disciplinas de preferência 
			e área de formação para que o coordenador possa montar a grade de aulas de acordo com os dados que informei)].
		\end{itemize}
		
			
		\subsection{Diagramas de Caso de Uso}
		\begin{figure}[h]
			\begin{center}
				 \includegraphics[width=450px]{CasoUsoDisponibilidadeHorario}
				 \caption{Caso de Uso - Disponibilidade de Horário}
				 \label{fig:CasoUsoDisponibilidadeHorario}
			\end{center}
		\end{figure}
		
		
		\begin{figure}[h]
					\begin{center}
						 \includegraphics[width=450px]{casoUsoInformarHorarioDisponivelAula}
						 \caption{Caso de Uso Expandido - Informar Disponibilidade de Horário}
						 %\label{fig:CasoUsoDisponibilidadeHorario}
					\end{center}
		\end{figure}
		
		\begin{figure}[h]
							\begin{center}
								 \includegraphics[width=450px]{casoUsoVisualizarHorarioDisponivelAula}
								 \caption{Caso de Uso Expandido - Visualizar Disponibilidade de Horário}
								 %\label{fig:CasoUsoDisponibilidadeHorario}
							\end{center}
				\end{figure}
				
		\begin{figure}[h]
									\begin{center}
										 \includegraphics[width=450px]{casoUsoRemoverHorarioDisponivelAula}
									 \caption{Caso de Uso Expandido - Remover Disponibilidade de Horário}
									 %\label{fig:CasoUsoDisponibilidadeHorario}
								\end{center}
					\end{figure}
		
		\FloatBarrier
		\begin{landscape}
		\subsection{Diagramas de Sequência}
						\begin{figure}[h]
							\begin{center}
							\includegraphics[width=500px]{SequenceDisponibilidade}
							\caption{Diagrama de Sequência}
							\label{fig:SequenceDisponibilidade}
							\end{center}				
						\end{figure}
						\FloatBarrier
		\end{landscape}
			
	\clearpage
	\section{User Story 3}

		\begin{itemize}
			\item Como professor, eu quero visualizar minha grade horária de aulas para saber os horários que tenho que dar aula.
			[Adicionado a Semana 03 (Como professor, eu quero informar minhas disciplinas de preferência e área de formação para que o coordenador 
			possa montar a grade de aulas de acordo com os dados que informei.)].
		\end{itemize}
		

		
		\subsection{Diagramas de Caso de Uso}
			\begin{figure}[h]
				\begin{center}
				 \includegraphics[width=450px]{CasoUsoUserStory3}
				 \caption{Diagrama de Caso de Uso - User Story 03}
				 \label{fig:CasoUsoUserStory3}
				\end{center}
			\end{figure}
			
			\begin{figure}[h]
				\begin{center}
				 \includegraphics[width=450px]{CasoUsoAreaFormacao}
				 \caption{Diagrama de Caso de Uso - Área de Formação}
				 \label{fig:CasoUsoAreaFormacao}
				\end{center}
			\end{figure}
			

		\begin{figure}[h]
			\begin{center}
				 \includegraphics[width=450px]{casoUsoVisualizarGradeAulas}
				 \caption{Caso de Uso Expandido - Visualizar Grade de Aulas}
				 %\label{fig:CasoUsoDisponibilidadeHorario}
			\end{center}
		\end{figure}
		

		\begin{figure}[h]
			\begin{center}
				 \includegraphics[width=450px]{casoUsoInformarNivelInteresseDisciplina}
				 \caption{Caso de Uso Expandido - Informar Nível de Interesse em Disciplina}
				 %\label{fig:CasoUsoDisponibilidadeHorario}
			\end{center}
		\end{figure}
		
\begin{figure}[h]
			\begin{center}
				 \includegraphics[width=450px]{casoUsoAdionarAreaFormacao}
				 \caption{Caso de Uso Expandido - Adicionar Área de Formação}
				 %\label{fig:CasoUsoDisponibilidadeHorario}
			\end{center}
		\end{figure}
		
\begin{figure}[h]
			\begin{center}
				 \includegraphics[width=450px]{casoUsoRemoverAreaFormacao}
				 \caption{Caso de Uso Expandido - Remover Área de Formação}
				 %\label{fig:CasoUsoDisponibilidadeHorario}
			\end{center}
		\end{figure}
		
\begin{figure}[h]
			\begin{center}
				 \includegraphics[width=450px]{casoUsoVisualizarAreaFormacao}
				 \caption{Caso de Uso Expandido - Visualizar Área de Formação}
				 %\label{fig:CasoUsoDisponibilidadeHorario}
			\end{center}
		\end{figure}
		
	\clearpage
	\begin{landscape}
			\subsection{Diagramas de Sequência}
			\begin{figure}[h]
				\begin{center}
			\includegraphics[width=650px]{SequenceGradeAulas}
			\caption{Diagrama de Sequência - Grade de Aulas}
			\label{fig:SequenceGradeAulas}
				\end{center}
				
			\end{figure}
			\FloatBarrier
			\begin{figure}[h]
					\begin{center}
				\includegraphics[width=650px]{SequenceInteresseDisciplina}
				\caption{Diagrama de Sequência - Interesse Disciplina}
				\label{fig:SequenceInteresseDisciplina}
					\end{center}
			\end{figure}
			\FloatBarrier	
		
	\clearpage
		\begin{figure}[h]
		\begin{center}
			\includegraphics[width=650px]{SequenceAreaFormacao}
			\caption{Diagrama de Sequência -  Área de Formação}
			\label{fig:SequenceAreaFormacao}
			\end{center}
		\end{figure}
		\FloatBarrier
	\end{landscape}	
	\clearpage
	
	
		\section{Diagrama de Classes}
		
			A figura \ref{fig:DiagramaClassesModelo} na próxima página mostra o diagrama de classes da solução implementada para as \emph{user stories} do \emph{sprint}. As classes e atributos referente ao sprint estão num tom vermelho.
			
	
		\section{Casos de Teste}
		
		 Por limitação de exportação das descrições da ferramenta, os casos de teste podem ser visto no link \url{http://chicago1.vsnetwork.net:8080/klaros-web/pages/login.seam} com usuário \emph{professor} e senha \emph{professor}
		 
		 Os casos de teste estão agrupados em 2 suites de teste:
		 \begin{itemize}
		 \item TS00005 - Preferências de Disciplinas / Areá de Formação / Grade Horários
		 \item TS00004 - Manter Horário Disponível
		 \end{itemize}
		 
		 \begin{landscape}
		 			\begin{figure}[htp]
		 				\begin{center}
		 					 \includegraphics[width=600px]{DiagramaClassesModelo}
		 					 \caption{Diagrama de Classes Modelo}
		 					 \label{fig:DiagramaClassesModelo}
		 				\end{center}
		 			\end{figure}
		 			%\FloatBarrier
		 		\end{landscape}
		 		 
\clearpage
	\section{Acesso ao Sistema}
			De acordo com o planejamento construído no primeiro \emph{sprint}, o qual definimos nossa infraestrutura de desenvolvimento, 
			o sistema esta em integração contínua com o servidor de testes on-line, contendo todas as funcionalidades da versão 1.0 agora incrementadas para a versão 2.0, 
			podendo ser acessado a partir da seguinte url:
		
			\url{http://chicago1.vsnetwork.net/}
			
			
			

	\section{Acesso às Ferramentas}
		
		O código fonte do projeto pode ser acessado em nosso repositório do github\cite{GITHUB} através da url: \url{https://github.com/dextervip/rpv}. 
		
		As construções automatizadas do jenkins podem ser acessadas através da url: \url{http://chicago2.vsnetwork.net:8080/job/RPV}.

	\clearpage

	%Referências Bibliograficas
	\nocite{*}
	\bibliographystyle{abnt-num}
	%\bibliographystyle{plain}		
	\bibliography{bibliografia}		

\end{document}