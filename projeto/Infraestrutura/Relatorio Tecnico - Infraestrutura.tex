\documentclass[12pt,a4paper]{article}
\usepackage[a4paper, inner=1.5cm, outer=2cm, top=3cm, bottom=2cm, bindingoffset=1cm]{geometry}
\usepackage[utf8]{inputenc}
\usepackage[english,brazilian]{babel}
\usepackage{hyperref}
\usepackage{url}
\begin{document}

\title{Infraestrutura}
\author{Rafael Tavares Amorim}
\date{31/03/2012}
\maketitle
\thispagestyle{empty}
\clearpage
\tableofcontents
\thispagestyle{empty}
\clearpage
\section{Ambiente de Desenvolvimento}
	\subsection{IDE}
		\subsubsection{Introdução}

				A IDE ou Ambiente de Desenvolvimento Integrado é um software que ajuda no
				processo de desenvolvimento de software. Com o objetivo de aumentar a
				produtividade, torna-se indispensável, tipicamente uma IDE contem as
				seguintes características
				\begin{itemize}
					\item Editor de código fonte
					\item Construtor de software
					\item Depurador de código
				\end{itemize}
				
				Dentre as IDE que estão atualmente sendo mais utilizadas, vamos falar sobre
				a Aptana, uma distribuição do Eclipse e Netbeans.
		
		\subsubsection{Aptana}
		
		\subsubsection{Netbeans}
		
		\subsubsection{Conclusão}


	\subsection{Ferramenta Case para UML}
		\subsubsection{Introdução}
		\subsubsection{ArgoUML}
		\subsubsection{Ferramenta Y}

	\subsection{Ferramenta Case para Banco de Dados}
		\subsubsection{Introdução}
		\subsubsection{PowerDesigner}
		\subsubsection{MySQL Workbench}
		
\clearpage
\section{Ambiente de Teste}
	\subsection{Ferramenta para Teste Automatizado}
		\subsubsection{Introdução}
		\subsubsection{Ferramenta X}
		\subsubsection{Ferramenta Y}
		
	\subsection{Ferramenta para Construção e Gerenciamento de Caso de Teste}
		\subsubsection{Introdução}
		\subsubsection{Testlink}
		\subsubsection{Tarantula}
		\subsubsection{Klaros}
		
	\subsection{Ferramenta para Gerenciamento de Defeitos}
		\subsubsection{Introdução}
		\subsubsection{Mantis}
		\subsubsection{Redmine}

\clearpage		
\section{Ambiente de Gerenciamento}
	\subsection{Construção de Software}
		\subsubsection{Introdução}
		\subsubsection{Jenkins}
		\subsubsection{phpUnderControl}
		
		\begin{table}[h]
			\tabcolsep=0.11cm
			\begin{tabular}{|c|c|c|c|c|c|c|c|}
			\hline Nome 			& Licença 	& Linguagem & Tipo de Construção 	& Recurso 2 & Recurso 3 & Recurso 4  \\
			\hline  CruiseControl	&  			&  			& Automática			&  			&  			&  			\\
			\hline  Jenkins			& MIT		& Java		& Automática			&  			&  			&  			\\
			\hline  phpUnderControl	&  			&  			& Automática 			&  			&  			&  			\\ 
			\hline  Travis			&  			&  			& Automática			&  			&  			&  			\\ 
			\hline  Ant				&  			&  			& Manual				&  			&  			&  			\\ 
			\hline  Maven			&  			&  			& Manual				&  			&  			&  			\\ 
			\hline  Phing			&  			& PHP		& Manual				&  			&  			&  			\\
			\hline  
			\end{tabular} 
			\caption{Comparação entre ferramentas de construção de software}
		\end{table}
		
	\subsection{Controle de Versão}
		\subsubsection{Introdução}
		\subsubsection{SVN}
		\subsubsection{Git}
		
	\subsection{Gerenciamento de Tarefas}
		\subsubsection{Introdução}
		\subsubsection{Ferramenta X}
		\subsubsection{Ferramenta Y}
	
	\subsection{Gerenciamento de Mudança}
		\subsubsection{Introdução}
		\subsubsection{Ferramenta X}
		\subsubsection{Ferramenta Y}
		
		

\clearpage
%Referências Bibliograficas
\nocite{*}
\bibliographystyle{plain}		
\bibliography{bibliografia}		

\end{document}