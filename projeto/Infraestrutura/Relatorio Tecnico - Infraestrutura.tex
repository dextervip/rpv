\documentclass[12pt,a4paper]{article}
\usepackage[a4paper, inner=1.5cm, outer=2cm, top=3cm, bottom=2cm, bindingoffset=1cm]{geometry}
\usepackage[utf8]{inputenc}
\usepackage[english,brazilian]{babel}
\usepackage{hyperref}
\usepackage{url}
\begin{document}

\title{Infraestrutura}
\author{
Rafael Tavares Amorim \\
Juliano R. Macedo \\
Maria Elizabeth \\
Wolmir 
}
\date{31/03/2012}
\maketitle
\thispagestyle{empty}
\clearpage
\tableofcontents
\thispagestyle{empty}
\clearpage
\section{Ambiente de Desenvolvimento}
	\subsection{IDE}
		\subsubsection{Introdução}

				A IDE ou Ambiente de Desenvolvimento Integrado é um software que ajuda no
				processo de desenvolvimento de software. Com o objetivo de aumentar a
				produtividade, torna-se indispensável, tipicamente uma IDE contem as
				seguintes características
				\begin{itemize}
					\item Editor de código fonte
					\item Construtor de software
					\item Depurador de código
				\end{itemize}
				
				Dentre as IDE que estão atualmente sendo mais utilizadas, vamos falar sobre
				a Aptana, uma distribuição do Eclipse e Netbeans.
		
		\subsubsection{Aptana}
		
		\subsubsection{Netbeans}
		
		\subsubsection{Conclusão}


	\subsection{Ferramenta CASE para UML}
		\subsubsection{Introdução}
		\subsubsection{ArgoUML}
		\subsubsection{Ferramenta Y}

	\subsection{Ferramenta CASE para Banco de Dados}
		\subsubsection{Introdução}
			Uma ferramenta CASE para banco de dados auxilia no processo de projeto de banco de bancos. Um projeto de banco de
			dados é constituido de duas fases: Modelagem Conceitual e Projeto Lógico. Na primeira fase, é realizado um projeto do
			modelo de banco de dados em forma de diagrama ER(Entidade Relacionamento), que descreve os tipos de informações que o
			banco de dados irá armazenar. Enquanto na segunda fase, trata de transformar o modelo conceitual em um modelo lógico
			que define como será implementado em SGBD específico.
			
			Para que seja considerada uma ferramenta CASE para banco de dados, a ferramenta deve auxiliar de forma que permita
			todo o projeto de banco de dados ser inicialmente feito por diagramas e auxiliar na transformação de conceitual para
			lógico. E por transformando em um esquema de banco de dados físico.
		\subsubsection{PowerDesigner}
		\subsubsection{MySQL Workbench}
		
\clearpage
\section{Ambiente de Teste}
	\subsection{Ferramenta para Teste Automatizado}
		\subsubsection{Introdução}
		\subsubsection{Ferramenta X}
		\subsubsection{Ferramenta Y}
		
	\subsection{Ferramenta para Construção e Gerenciamento de Caso de Teste}
		\subsubsection{Introdução}
		\subsubsection{Testlink}
		\subsubsection{Tarantula}
		\subsubsection{Klaros}
		
	\subsection{Ferramenta para Gerenciamento de Defeitos}
		\subsubsection{Introdução}
		
			A gestão de Defeitos permite a equipe de testes acompanhar a qualidade do software, 
			de acordo com os registros de bugs, acompanhando suas correções e o surgimento de outros defeitos. 
			Para executar-se uma gestão eficiente, podemos utilizar algumas ferramentas que permitem este acompanhamento de forma otimizada, 
			a baixo falaremos sobre as duas mais conceituadas do mercado.
			
		\subsubsection{Mantis}
		
			Mantis é uma ferramenta para rastreamento e mapeamento de erros em um sistema de software, 
			apresenta grande popularidade e bom conceito no mercado, ela é Open Source e roda sobre plataforma web, 
			sendo compatível com a maioria dos navegadores existentes. 	

				\begin{itemize}
					\item Vantagens: 
						Ambiente leve e prático, apresenta como ponto forte sua tradição, estando presente no mercado a mais de dez anos, 
					  esta disponível em mais de quinze idiomas e permite a integração de diversos plugins, tornando o ambiente de trabalho flexível
					  com forme as necessidades da equipe. 					
					
					\item Desvantagens:
						Interface confusa e com pouca usabilidade, comumente não é recomendada a sua utilização em grandes projetos, 
						devido ao seu gerenciamento de defeitos não ser muito eficiente e não permite a edição prática das informações dentro dele. 
						
				\end{itemize}
						
		\subsubsection{Redmine}
		
			 Ferramenta de Gerenciamento de projetos, extremamente conceituada no mercado, contem diversos recursos, 
			 entre os mais utilizados esta o poderoso gerenciamento de bugs, o qual permite ao usuário acompanhar o 
			 andamento da correção e/ou surgimento de novos defeitos por meio dos menus: Atividades, Planejamento, Tarefas, Wiki e Fóruns.
			 O Redmine é escreto em Ruby on Rails, é multiplataforma e Open Source, trabalhando sobre os termos da licença GNU General Public License v2 (GPL).
			 
			 \begin{itemize}
			 	\item Vatangens:
			 		Ambiente prático e simples, interface clara e eficiente, excelente organização das informações inseridas dentro da ferramenta. Permite a aglutinação de informações referentes ao projeto em desenvolvimento, permitindo assim que as equipes de trabalho  concentrem suas tarefas, analises, resultados, etc, em um mesmo ambiente. 
			 	
			 	\item Desvantgens:
			 		Como qualquer ferramenta o Redmine sozinho não garante um ambiente produtivo ou eficiente, para se obter o melhor proveito dos recursos disponíveis neste software, é necessário empregar regras e padrões de utilização, para todos os membros do grupo, visando a melhor organização e clareza possível nas informações inseridas no ambiente. Isso é valido para o gerenciamento de defeitos, é necessário definir padrões para o reporte e correção de cada bug encontrado, evitando-se a duplicação e ambiguidade das informações. 
			 	
			 \end{itemize}
			 
		
		\subsubsection{Conclusão}
		
			Apos analise criteriosa de cada opção, a ferramenta escolhida foi o Redmine. 
			Apesar de termos aqui opções muito conceituadas como o proprío Mantis, o fator decisivo foi a praticidade em trabalhar-se com o Redmine, 
			pois ele é um aglutinador de recursos, ou seja, dentro de um projeto não é necessário trabalhar com outro software específico para o gerenciamento de defeitos, 
			sendo que o Redmine nos permite utilizar suas características para realizarmos este gerecimento, ou seja, a equipe terá o que necessita em mãos. 
			Isso representa ganho de tempo e eficiência, pois em apenas um ambiente de gerenciamento a equipe pode trabalhar sobre diversos aspectos/problemas, 
			gerando uma aglutinação dos dados referentes ao projeto.
		
				

\clearpage		
\section{Ambiente de Gerenciamento}
	\subsection{Construção de Software}
		\subsubsection{Introdução}
		\subsubsection{Ant}
		\subsubsection{Maven}
		\subsubsection{Phing}
		\subsubsection{Jenkins}
		\subsubsection{phpUnderControl}
		
		\begin{table}[h]
			\tabcolsep=0.11cm
			\begin{tabular}{|c|c|c|c|c|c|c|c|}
			\hline Nome 			& Licença 	& Linguagem & Tipo de Construção 	& Recurso 2 & Recurso 3 & Recurso 4  \\
			\hline  CruiseControl	&  			&  			& Automática			&  			&  			&  			\\
			\hline  Jenkins			& MIT		& Java		& Automática			&  			&  			&  			\\
			\hline  phpUnderControl	&  			&  			& Automática 			&  			&  			&  			\\ 
			\hline  Travis			&  			&  			& Automática			&  			&  			&  			\\ 
			\hline  Ant				&  			&  			& Manual				&  			&  			&  			\\ 
			\hline  Maven			&  			&  			& Manual				&  			&  			&  			\\ 
			\hline  Phing			&  			& PHP		& Manual				&  			&  			&  			\\
			\hline  
			\end{tabular} 
			\caption{Comparação entre ferramentas de construção de software}
		\end{table}
		
	\subsection{Controle de Versão}
		\subsubsection{Introdução}
		\subsubsection{SVN}
		\subsubsection{Git}
		
	\subsection{Gerenciamento de Tarefas}
		\subsubsection{Introdução}
		\subsubsection{Ferramenta X}
		\subsubsection{Ferramenta Y}
	
	\subsection{Gerenciamento de Mudança}
		\subsubsection{Introdução}
		\subsubsection{Ferramenta X}
		\subsubsection{Ferramenta Y}
		
		

\clearpage
%Referências Bibliograficas
\nocite{*}
\bibliographystyle{plain}		
\bibliography{bibliografia}		

\end{document}