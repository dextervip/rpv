\documentclass[12pt,a4paper]{article}
\usepackage[a4paper, inner=1.5cm, outer=2cm, top=3cm, bottom=2cm, bindingoffset=1cm]{geometry}
\usepackage[utf8]{inputenc}
\usepackage[english,brazilian]{babel}
\usepackage{hyperref}
\usepackage{url}
\begin{document}

\title{Infraestrutura}
\author{Rafael Tavares Amorim}
\date{31/03/2012}
\maketitle
\thispagestyle{empty}
\newpage
\tableofcontents
\thispagestyle{empty}
\newpage
\section{Ambiente de Desenvolvimento}
	\subsection{IDE}
		\subsubsection{Introdução}

				A IDE ou Ambiente de Desenvolvimento Integrado é um software que ajuda no
				processo de desenvolvimento de software. Com o objetivo de aumentar a
				produtividade, torna-se indispensável, tipicamente uma IDE contem as
				seguintes características
				\begin{itemize}
					\item Editor de código fonte
					\item Construtor de software
					\item Depurador de código
				\end{itemize}
				
				Dentre as IDE que estão atualmente sendo mais utilizadas, vamos falar sobre
				a Aptana, uma distribuição do Eclipse e Netbeans.
				

		\subsubsection{Objetivo}


	\subsection{Ferramenta Case para UML}
	\subsection{Ferramenta Case para Banco de Dados}

\section{Ambiente de Teste}
	\subsection{Ferramenta para Teste Automatizado}
	\subsection{Ferramenta para Construção de Caso de Teste}
	\subsection{Ferramenta para Gerenciamento de Caso de Teste}
	\subsection{Ferramenta para Gerenciamento de Defeitos}
\section{Ambiente de Gerenciamento}
	\subsection{Construção de Software}
		\subsubsection{Introdução}
		\subsubsection{Jenkins}
		\subsubsection{phpUnderControl}
	\subsection{Controle de Versão}
	\subsection{Gerenciamento de Tarefas}
	\subsection{Gerenciamento de Mudança}
		
		\begin{table}[h]
			\tabcolsep=0.11cm
			\begin{tabular}{|c|c|c|c|c|c|c|c|}
			\hline Nome 			& Licença 	& Linguagem & Recurso 1 & Recurso 2 & Recurso 3 & Recurso 4 & Recurso 5 \\ 
			\hline  CruiseControl	&  			&  			&  			&  			&  			&  			&  			\\ 
			\hline  Jenkins			& MIT		& Java		&  			&  			&  			&  			&  			\\ 
			\hline  phpUnderControl	&  			&  			&  			&  			&  			&  			&  			\\ 
			\hline  Travis			&  			&  			&  			&  			&  			&  			&  			\\ 
			\hline  				&  			&  			&  			&  			&  			&  			&  			\\ 
			\hline  				&  			&  			&  			&  			&  			&  			&  			\\ 
			\hline 
			\end{tabular} 
			\caption{Comparação entre ferramentas de construção de software}
		\end{table}
%\cite{Test}
\clearpage
%Referências Bibliograficas
\nocite{*}
\bibliographystyle{plain}		
\bibliography{bibliografia}		

\end{document}