\documentclass[12pt,a4paper]{article}
\usepackage[a4paper, inner=1.5cm, outer=2cm, top=3cm, bottom=2cm, bindingoffset=1cm]{geometry}
\usepackage[utf8]{inputenc}
\usepackage[english,brazilian]{babel}
\usepackage{hyperref}
\usepackage{url}
\begin{document}

\title{Infraestrutura de Desenvolvimento}
\author{
Rafael Tavares Amorim \\
Juliano R. Macedo \\
Maria Elizabeth \\
Wolmir 
}
\date{31/03/2012}
\maketitle
\thispagestyle{empty}
\clearpage
\tableofcontents
\thispagestyle{empty}
\clearpage
\section{Ambiente de Desenvolvimento}
	\subsection{IDE}
		\subsubsection{Introdução}

				A IDE ou Ambiente de Desenvolvimento Integrado é um software que ajuda no
				processo de desenvolvimento de software. Com o objetivo de aumentar a
				produtividade, torna-se indispensável, tipicamente uma IDE contem as
				seguintes características
				\begin{itemize}
					\item Editor de código fonte
					\item Construtor de software
					\item Depurador de código
				\end{itemize}
				
				Dentre as IDE que estão atualmente sendo mais utilizadas, vamos falar sobre
				a Aptana, uma distribuição do Eclipse e Netbeans.
		
		\subsubsection{Aptana}
		
		\subsubsection{Netbeans}
		
		\subsubsection{Conclusão}


	\subsection{Ferramenta CASE para UML}
		\subsubsection{Introdução}
		\subsubsection{ArgoUML}
		\subsubsection{Ferramenta Y}

	\subsection{Ferramenta CASE para Banco de Dados}
		\subsubsection{Introdução}
			Uma ferramenta CASE para banco de dados auxilia no processo de projeto de banco de bancos. Um projeto de banco de
			dados é constituido de duas fases: Modelagem Conceitual e Projeto Lógico. 
			
			Na primeira fase, é realizado um projeto do modelo de banco de dados em forma de diagrama ER(Entidade
			Relacionamento), que descreve os tipos de informações que o banco de dados irá armazenar. Enquanto na segunda fase,
			trata de transformar o modelo conceitual em um modelo lógico que define como será implementado em SGBD específico.
			
			Para que seja considerada uma ferramenta CASE para banco de dados, a ferramenta deve auxiliar de forma que permita
			todo o projeto de banco de dados ser inicialmente feito por diagramas e auxiliar na transformação de conceitual para
			lógico. E por transformando em um esquema de banco de dados físico.
			
		\subsubsection{PowerDesigner}
			O PowerDesigner é uma ferramenta CASE Empresarial para o processo de desenvolvimento de software. Atualmente é
			mantido pela empresa americana Sybase, roda nativamente em computadores com o sistema operacional Microsoft Windows
			e sobre o eclipse através de plugin.
			
			O PowerDesginer? possui destaque no mercado de arquitetura de dados sendo umas das ferramentas mais utilizadas em
			organizações, universidades e grandes empresas. Dentre outras funcionalidades para banco de dados podem ser
			destacadas:
			\begin{itemize}
			  \item Engenharia reversa de Bancos
			  \item Controle de vrsão
			  \item Organização/disposição de componentes por Layout
			  \item Rotas automáticas entre ligações de componentes
			  \item Checagem de consistência do modelo
			  \item
			\end{itemize}
			
			Para modelagem de banco de dados, é oferecida varias notações como Barker, Entity Relationship, Merise entre outras.
			Sendo divida em três fases: Modelagem conceitual, logica e física.

			Na modelagem conceitual temos a possibilidade de criar entidades, atributos e relacionamentos. Na segunda fase da
			modelagem logica fazemos as definições de índices e especificação do nome das chaves estrangeiras entre outras. E na
			última fase de modelagem física pode se definir views, triggers, procedures e outras configurações dependendo do SGBD.

			O PowerDesginer? mostra-se uma ótima ferramenta para modelagem de banco de dados, possuindo diversos recursos que
			irão aumentar a produtividade e organização.
			
		\subsubsection{MySQL Workbench}
		
			Em linhas gerais podemos dizer que o MySQL Workbench é uma ferramenta que possibilita um DBA ou desenvolvedores em
			geral, criarem e gerenciarem de forma visual todos os tipos de bases de dados MySQL, desde uma base utilizada em uma
			simples aplicação web, até bases mais complexas, como as existentes em sistemas de data warehouses ou OLTP. Destaques
			da ferramenta:
			
			\begin{itemize}
			  \item Criação/Edição visual da base de dados: Acelera o desenvolvimento, diminuindo a chance de erros em tarefas
			  comuns. Oferece validação de schemas para o modelo criado, auxiliando na utilização das melhores práticas de
			  modelagem de dados e evitando erros na construção de diagramas ER ou criação da base fisicamente.
			  \item Forward Engineering e Engenharia Reversa: Possibilita a criação de um modelo completo a partir de um script
			  SQL ou de uma base de dados existente, incluindo o diagrama ER com as tabelas pré-dispostas no mesmo. Oferece
			  também a construção do banco de dados fisicamente através da modelagem visual.
			  \item Rastreamento e gerenciamento de mudanças: MySQL Workbench inclui as funcionalidades de Sincronização de
			  Schemas e utilitários de comparação. Por exemplo, um DBA pode comparar duas bases ativas ou comparar uma modelagem
			  feita no MySQL Workbench e uma base existente, visualizando as diferenças, com a possibilidade de sincronização
			  entre um modelo e uma base ou vice versa.
			\end{itemize}

			Documentação: O MySQL Workbench inclui o DBDoc que auxilia no processo de documentação dos modelos e objetos
			existentes. Esta documentação pode apresentar-se em texto puro ou no formato HTML.
		
\clearpage
\section{Ambiente de Teste}
	\subsection{Ferramenta para Teste Automatizado}
		\subsubsection{Introdução}
		\subsubsection{Ferramenta X}
		\subsubsection{Ferramenta Y}
		
	\subsection{Ferramenta para Construção e Gerenciamento de Caso de Teste}
		\subsubsection{Introdução}
		\subsubsection{Testlink}
		\subsubsection{Tarantula}
		\subsubsection{Klaros Testmanagement}
			É uma aplicação web baseada em AJAX escrita em Java para organização e gerenciamento do processo de teste. O sistema
			permite a construção dos casos de teste, gerenciar casos de teste, dividir em suítes de testes, cadastrar os
			ambientes de teste e sistemas que estão rodando o teste, executar testes e coletar os resultados.
			
			Pelos testes realizados no sistema, temos a possibilidade da construção de casos de teste, os casos de teste são
			nomeados, descritos (descrição, pré-condição e pós-condição), priorizados entre os demais, possui estado de
			execução, modo de execução e podem ser divididos em vários passos. Também se pode definir o ambiente de execução
			como Sistema Operacional ou Navegador para aplicações Web. A interface do sistema é bem amigável, não houve
			dificuldades de navegação sobre a aplicação.
			
			\begin{itemize}
			  \item Possui versionamento e histórico sobre os dados.
			  \item Trabalha com casos de teste e seus resultados.
			  \item Gera relatórios em gráficos e tabela sobre os testes. Podendo ser exportado em vários formatos como PDF,
			  HTML, entre outros.
			  \item Possui suporte de integração com as ferramentas de gerenciamento de projeto Redmine, Jira, Trac e Bugzilla.
			  \item É possível através de plugin dos sistemas de integração continua (Hudson/Jenkins) importar os relatórios de
			  execução dos testes automatizados.
			  \item Pode ser instalado em Windows por meio de um executável e Linux por meio de um arquivo jar.
			\end{itemize}
			
			A ferramenta se mostrou muito fácil de instalar como um software que você instala no dia a dia com poucos cliques,
			possui um instalador que facilita muito e instalar todas as dependencias e o servidor em si. Após a instalação, a
			configuração nada mais foi do que informar o usuário e senha para o administrador assim podendo-se inciar as
			atividades de teste na ferramenta. Para integração com gerenciamento de defeitos, na parte administrativa, é
			informado o gerenciamento de defeito, endereço, usuário e senha. 
		
	\subsection{Ferramenta para Gerenciamento de Defeitos}
		\subsubsection{Introdução}
		
			A gestão de Defeitos permite a equipe de testes acompanhar a qualidade do software, 
			de acordo com os registros de bugs, acompanhando suas correções e o surgimento de outros defeitos. 
			Para executar-se uma gestão eficiente, podemos utilizar algumas ferramentas que permitem este acompanhamento de forma otimizada, 
			a baixo falaremos sobre as duas mais conceituadas do mercado.
			
		\subsubsection{Mantis}
		
			Mantis é uma ferramenta para rastreamento e mapeamento de erros em um sistema de software, 
			apresenta grande popularidade e bom conceito no mercado, ela é Open Source e roda sobre plataforma web, 
			sendo compatível com a maioria dos navegadores existentes. 	

				\begin{itemize}
					\item Vantagens: 
						Ambiente leve e prático, apresenta como ponto forte sua tradição, estando presente no mercado a mais de dez anos, 
					  esta disponível em mais de quinze idiomas e permite a integração de diversos plugins, tornando o ambiente de trabalho flexível
					  com forme as necessidades da equipe. 					
					
					\item Desvantagens:
						Interface confusa e com pouca usabilidade, comumente não é recomendada a sua utilização em grandes projetos, 
						devido ao seu gerenciamento de defeitos não ser muito eficiente e não permite a edição prática das informações dentro dele. 
						
				\end{itemize}
						
		\subsubsection{Redmine}
		
			 Ferramenta de Gerenciamento de projetos, extremamente conceituada no mercado, contem diversos recursos, 
			 entre os mais utilizados esta o poderoso gerenciamento de bugs, o qual permite ao usuário acompanhar o 
			 andamento da correção e/ou surgimento de novos defeitos por meio dos menus: Atividades, Planejamento, Tarefas, Wiki e Fóruns.
			 O Redmine é escrito em Ruby on Rails, é multiplataforma e Open Source, trabalhando sobre os termos da licença GNU General Public License v2 (GPL).
			 
			 \begin{itemize}
			 	\item Vatangens:
			 		Ambiente prático e simples, interface clara e eficiente, excelente organização das informações inseridas dentro da ferramenta. Permite a aglutinação de informações referentes ao projeto em desenvolvimento, desta formar as equipes de trabalho  concentram suas tarefas, analises, resultados e etc, em um mesmo ambiente. 
			 	
			 	\item Desvantgens:
			 		Como qualquer ferramenta o Redmine sozinho não garante um ambiente produtivo ou eficiente, para se obter o melhor proveito dos recursos disponíveis neste software, é necessário empregar regras e padrões de utilização, para todos os membros do grupo, visando a melhor organização e clareza possível nas informações inseridas no ambiente. Isso é valido para o gerenciamento de defeitos, é necessário definir padrões para o reporte e correção de cada bug encontrado, evitando-se a duplicação e ambiguidade das informações. 
			 	
			 \end{itemize}
			 
		
		\subsubsection{Conclusão}
		
			Apos analise criteriosa de cada opção, a ferramenta escolhida foi o Redmine. 
			Apesar de termos aqui opções muito conceituadas como o proprío Mantis, o fator decisivo foi a praticidade em trabalhar-se com o Redmine, 
			pois ele é um aglutinador de recursos, ou seja, dentro de um projeto não é necessário trabalhar com outro software específico para o gerenciamento de defeitos, 
			sendo que o Redmine nos permite utilizar suas características para realizarmos este gerecimento, ou seja, a equipe terá o que necessita em mãos. 
			Isso representa ganho de tempo e eficiência, pois em apenas um ambiente de gerenciamento a equipe pode trabalhar sobre diversos aspectos/problemas, 
			gerando uma aglutinação dos dados referentes ao projeto.
		
				

\clearpage		
\section{Ambiente de Gerenciamento}
	\subsection{Construção de Software}
		\subsubsection{Introdução}
		\subsubsection{Ant}
		\subsubsection{Maven}
		\subsubsection{Phing}
		\subsubsection{Jenkins}
		\subsubsection{phpUnderControl}
		
		\begin{table}[h]
			\tabcolsep=0.11cm
			\begin{tabular}{|c|c|c|c|c|c|c|c|}
			\hline Nome 			& Licença 	& Linguagem & Tipo de Construção 	& Recurso 2 & Recurso 3 & Recurso 4  \\
			\hline  CruiseControl	&  			&  			& Automática			&  			&  			&  			\\
			\hline  Jenkins			& MIT		& Java		& Automática			&  			&  			&  			\\
			\hline  phpUnderControl	&  			&  			& Automática 			&  			&  			&  			\\ 
			\hline  Travis			&  			&  			& Automática			&  			&  			&  			\\ 
			\hline  Ant				&  			&  			& Manual				&  			&  			&  			\\ 
			\hline  Maven			&  			&  			& Manual				&  			&  			&  			\\ 
			\hline  Phing			&  			& PHP		& Manual				&  			&  			&  			\\
			\hline  
			\end{tabular} 
			\caption{Comparação entre ferramentas de construção de software}
		\end{table}
		
	\subsection{Controle de Versão}
		\subsubsection{Introdução}
		\subsubsection{SVN}
		\subsubsection{Git}
		
	\subsection{Gerenciamento de Tarefas}
		\subsubsection{Introdução}
			O gerenciamento de tarefas dentro de um projeto de software permite a equipe registrar e manipular todo e qualquer evento referente a ele, de forma simples e eficiente, evitando a desorganização destas atividades. 
		
		
		\subsubsection{Redmine}
			
			Ferramenta de Gerenciamento de projetos, extremamente conceituada no mercado, contem diversos recursos, entre os mais utilizados esta seu poderoso gerenciamento de tarefas e, o qual permite ao usuário acompanhar o andamento da  execução das tarefas, especialmente, por meio do menu: Tarefas.
			O Redmine é escrito em Ruby on Rails, multiplataforma e Open Source, trabalhando sobre os termos da licença GNU General Public License v2 (GPL v2).
		
		\begin{itemize}
			\item Vantagens: 
			Ambiente prático e simples, interface clara e eficiente, excelente organização das informações inseridas dentro da ferramenta. Permite a aglutinação de informações referentes ao projeto em desenvolvimento, permitindo assim que as equipes de trabalho, concentrem suas tarefas, analises, resultados, etc, em um mesmo ambiente. 
			Redmine é a o gestor de projetos open source mais famoso do mundo.
			O gestor de Tarefas do Redmine permite aos usuários organizar as atividades por: Prioridade, Categoria, Situação, Tipo, Versão, Porcentagem de Andamento, etc; E claro, o administrador geral do sistema(do Redmine) pode customizar os campos conforme sua vontade, adequando o software e suas ferramentas de acordo com as necessidade do projeto, isso permite uma flexibilidade dentro do ambiente de trabalho.
			
			
			\item Desvantagens: 
			Como qualquer ferramenta o Redmine sozinho não garante um ambiente produtivo ou eficiente, para se obter o melhor proveito dos recursos disponíveis neste software, é necessário empregar regras e padrões de utilização, para todos os membros do grupo, visando a melhor organização e clareza possível nas informações inseridas no ambiente. Isso é valido para o gerenciamento de defeitos, é necessário definir padrões para o reporte e correção de cada bug encontrado, evitando-se a duplicação e ambiguidade das informações. 
				
		\end{itemize}
		
		\subsubsection{Trac}
		
			Trac é um gerenciador de projetos e bug-tracker, ele é open-source, escrito em Python e multiplataforma. Trac trabalha sobre uma abordagem minimalista que permite ao usuário configurar seu ambiente conforme suas necessidades. 
		
		\begin{itemize}
			\item Vantagens: 
				Famoso gestor de projetos e conhecido no mercado, contem gerenciamento de tarefas, é baseado em hiperlinks e apresenta integração com controles de versões, como Subversion, Git, Mercurial, etc. Apresenta uma interface simples e fácilmente pode ser gerenciado pelo administrador do sistema. 
		
			\item Desvantagens: 
				Interface um pouco confusão, e por vezes "muito simples", pode comprometer a qualidade do trabalho desenvolvido dentro da ferramenta.
				E outro fator relevante, o grupo não conhece a ferramenta, e princialmente por já trabalharmos de modo integrado com o Redmine, dentro dos outros ambientes (teste e desenvolvimento), resolvemos não aprofundar as pesquisas dentro deste ferramenta.
			
		\end{itemize}
		
		
		\subsubsection{Conclusão}
			Inegavelmente por já trabalharmos com o Redmine à algum tempo e principalmente por já te-lo escolhido como plataforma de outros aspectos, como bug-tracker, e claro por ser um aglutinador de funções, decidimos de modo unânime por sua utilização como ferramenta de Gerenciamento de Tarefas. 
		
		
	
	\subsection{Gerenciamento de Mudança}
		\subsubsection{Introdução}
		\subsubsection{Ferramenta X}
		\subsubsection{Ferramenta Y}
		
		

\clearpage
%Referências Bibliograficas
\nocite{*}
\bibliographystyle{plain}		
\bibliography{bibliografia}		

\end{document}